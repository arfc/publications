% Plain Cover Letter
% LaTeX Template
%
% This template has been downloaded from:
% http://www.latextemplates.com
%
% Original author:
% Rensselaer Polytechnic Institute (http://www.rpi.edu/dept/arc/training/latex/resumes/)
%
%%%%%%%%%%%%%%%%%%%%%%%%%%%%%%%%%%%%%%%%%

%----------------------------------------------------------------------------------------
%       PACKAGES AND OTHER DOCUMENT CONFIGURATIONS
%----------------------------------------------------------------------------------------

\documentclass[10pt]{letter} % Default font size of the document, change to 10pt to fit more text
\usepackage{graphicx}
%\usepackage{newcent} % Default font is the New Century Schoolbook PostScript font
%\usepackage{helvet} % Uncomment this (while commenting the above line) to use the Helvetica font

% Margins
\topmargin=-1in % Moves the top of the document 1 inch above the default
\textheight=10in % Total height of the text on the page before text goes on to the next page, this can be increased in a longer letter
%\oddsidemargin=-10pt % Position of the left margin, can be negative or positive if you want more or less room
%\textwidth=6.5in % Total width of the text, increase this if the left margin was decreased and vice-versa

%\let\raggedleft\raggedright % Pushes the date (at the top) to the left, comment this line to have the date on the right
%\usepackage{wallpaper}
%\ULCornerWallPaper{0.97}{bk_logo.eps}


\begin{document}

%----------------------------------------------------------------------------------------
%       ADDRESSEE SECTION
%----------------------------------------------------------------------------------------

\begin{letter}{Lynn Weaver, Mostafa Ghiaasiaan, Imre Pazsit\\
Editors\\
Annals of Nuclear Energy}

%----------------------------------------------------------------------------------------
%       YOUR NAME & ADDRESS SECTION
%----------------------------------------------------------------------------------------

\address{Department of Nuclear, Plasma, and Radiological Engineering\\
Talbot Hall, Room 226\\
104 S Wright St\\
University of Illinois, Urbana-Champaign\\
Urbana, IL 61801}

\name{\vspace{5mm}
Alexander Lindsay\\
Postdoctoral Scholar}

%----------------------------------------------------------------------------------------
%       LETTER CONTENT SECTION
%----------------------------------------------------------------------------------------

\opening{Dear Professors Weaver, Ghiaasiann, Pazsit:}

Please find enclosed a manuscript entitled: ``Introduction to Moltres: an
Application for Simulation of Molten Salt Reactors'' which I am submitting for
exclusive consideration of publication as a research article in Annals of
Nuclear Energy.

This manuscript describes the new molten salt simulation package Moltres, built
on top of the Multiphysics Object-Oriented Simulation Environment (MOOSE). In
its current version, Moltres combines neutron diffusion, heat transport, and
precursor drift for simulation of thermal molten-salt reactors in two and three
dimensions. Through its MOOSE infrastructure, Moltres is inherently parallel; to
date simulations have been run on as many as 640 cpu cores. The enclosed
manuscript details application of Moltres for reproducing the predicted and
observed steady-state physics of the Molten Salt Reactor Experiment (MSRE)
conducted at Oak Ridge during the 1960s. Good qualitative agreement between
Moltres calculations and the MSRE design is demonstrated. Additionally, we show
nearly perfect parallel scaling of Moltres with increasing processor
count. Though this publication focuses on thermal molten salt reactors (MSRs), the code
in Moltres is general, and simulations of fast MSRs and liquid metal reactors
are planned in the upcoming months. 

Thank you for your consideration of our work.  I expect it will be of interest
to a broad readership concerned with multi-physics simulation of next generation
reactors.  Please address all correspondence concerning this manuscript to me
at alexlindsay239@gmail.com.

\closing{Sincere regards,}
%----------------------------------------------------------------------------------------

\end{letter}

\end{document}

