\documentclass[11pt]{article}

\usepackage[acronym,toc]{glossaries}
\include{acros}
\makeglossaries
\usepackage{fancyhdr}
\usepackage{pagecounting}
\usepackage[dvips]{color}
\usepackage{graphicx}
\usepackage{tikz}
\usetikzlibrary{arrows,positioning}
% Color Information from - http://www-h.eng.cam.ac.uk/help/tpl/textprocessing/latex_advanced/node13.html

% Trying to bold my name in the bib
\usepackage{xstring}
\def\FormatName#1{%
  \IfSubStr{#1}{Huff}{\textbf{#1}}{#1}%
}

\usepackage[left=1in, right=1in, top=1in, bottom=1in]{geometry}
\newcommand\bb[1]{\mbox{\em #1}}
\def\baselinestretch{1.1}
%\pagestyle{empty}
\newcommand{\hsp}{\hspace*{\parindent}}
\definecolor{gray}{rgb}{0.4,0.4,0.4}

\newcommand{\authorname}{Kathryn~D.~Huff }
\newcommand{\authoremail}{katyhuff@gmail.com}
\newcommand{\authorsite}{arfc.npre.illinois.edu}

\begin{document}

\pagestyle{fancy}
%\pagenumbering{gobble}
%\fancyhead[location]{text}
% Leave Left and Right Header empty.
%\lhead{}
%\rhead{}
\lhead{\textcolor{gray}{Investigator: Prof. \authorname}}
\rhead{\textcolor{gray}{Advanced Reactors and Fuel Cycles}}
%\rhead{\textcolor{gray}{\thepage/\totalpages{}}}
\renewcommand{\headrulewidth}{0pt}
\renewcommand{\footrulewidth}{0pt}
\fancyfoot[C]{\footnotesize \textcolor{gray}{\authorsite}}

\paragraph{The \gls{ARFC} group} pursues improved 
safety and sustainability of nuclear power. To do so, we model and simulate novel reactor designs
boasting inherent safety features (i.e. accident tolerant fuels or non-voiding 
coolants) and sustainable fuel cycles (i.e. high fuel utilization, online 
reprocessing strategies). These analyses rely on developing computational 
methods that navigate trade-offs between accuracy and compute time, with 
special treatment of coupled feedbacks among thermal-hydraulic phenomena, 
neutron transport, and fuel performance.
Simulations which faithfully capture this coupling at realistic spatial and
temporal resolution are only possible with the aid of high performance
computing resources such as the Blue Waters supercomputing facility at the 
University of Illinois in Urbana-Champaign.

\paragraph{The state of the art} in advanced nuclear reactor simulation (e.g. the 
CASL DOE innovation hub) is focused on traditional light water reactors.  The 
\gls{ARFC} group seeks to extend that state of the art by enabling similarly high 
fidelity simulation of more advanced reactor designs through development of 
models and tools applicable to their unique materials, geometries, and physics. 

\paragraph{Our current work} models the transient, three-dimensional, multi-scale physics
of promising molten salt reactors such as the \gls{MSRE} and \gls{MSBR}.
This work is well underway with our development of the Moltres application 
\cite{lindsay_moltres_2017}. Moltres relies on the \gls{MOOSE} framework 
\cite{gaston_moose:_2009} as well as its Navier-Stokes capabilities. These 
existing tools are extended with multi-group diffusion neutronics incorporating 
delayed neutron kinetics. We additionally support these calculations with few group 
constants which we have generated with Serpent 
\cite{leppanen_serpent_2012,leppanen_study_2015} and verfied using 
SCALE-NEWT \cite{nuclear_regulatory_commission_scale:_1997}.

Initial steady state simulations of the \gls{MSRE} have leveraged the Blue Waters 
supercomputer to establish successful coupling between neutronics and thermal 
hydraulics using the \gls{ARFC}-developed Moltres tool. 
These have initially been steady state simulations and have occupied up to many hundreds of nodes
simultaneously. Three dimensional coupling between neutronics and thermal hydraulics have been
established in the Moltres tool, but validation and verification efforts are
ongoing.


In the near term, these steady state calculations will provide the initial
conditions for full core reactor transient simulations. Such simulations
commonly occupy tens of thousands of CPU cores at a time and vary in completion
time. The MOOSE framework is shown to scale very well up to 10,000 cores 
\cite{gaston_moose:_2009}.
Transient and multi-scale simulations, which require greater capability per
simulation, may occupy up to 100,000 CPU cores at a time. Such transient
simulations will evaluate the safety performance of advanced reactors in severe
(beyond design basis) accidents in these reactors. These will seek to explore 
parameters that might increase \gls{LOFC}, \gls{LOHS}, and \gls{RIA} 
survivability in novel reactors.

\paragraph{The Blue Waters Supercomputer} (notably, the fastest supercomputer 
on a university campus \cite{ncsa_about_2017}).  is therefore essential to our 
efforts. The PI, Kathryn Huff, has been awarded a named Blue Waters Assistant 
Professorship in affiliation with the \gls{NCSA}, and has been accordingly 
allocated 100,000 node hours annually to support her work.  To solve these 
large systems of partial differential equations on a finite element mesh in a 
fully-coupled, implicit way, the MOOSE framework was designed to take advantage 
of exactly this kind of high performance computing capability.  The framework 
conducts fully implicit physics coupling with Newton-Krylov based solver 
methods and sophisticated automatic adaptive meshing.  These simulations are 
memory intensive, so the exceptional memory capability of the 
supercomputer is essential to performant simulation times. 

It is also important to note that rendering visualizations of the results can 
be computationally intensive was well. Accordingly, the \gls{ARFC} group has 
established a collaboration with the Data Exploration Laboratory 
\cite{turk_data_2017} in order to 
take advantage of HPC visualization capabilities of the yt software package 
when visualizing simulation results \cite{turk_yt:_2011}. 

\paragraph{The potential impact} of this work lies in the possibility of 
simultaneously detailed spatially and temporally resolved neutron flux and
temperature distributions. Such simulations can improve designs, help characterize
performance, inform reactor safety margins, and enable validation of numerical
modeling techniques for unique physics.

\paragraph{Our future work} will approach similarly challenging materials and
geometries such as those in sodium cooled, gas cooled, and very high
temperature reactor designs which also boast improved safety and
sustainability. Accordingly, potential areas of future collaboration include: 

\begin{itemize}
        \item validation collaborations between \gls{ARFC} and materials scientists, especially 
                experimentalists interested in high flux, high heat performance of 
                molten fluoride salts, liquid metals, graphites, silicon 
                carbides, and uranium nitrides
        \item verification benchmarking collaborations between \gls{ARFC} and computationalists 
                interested in comparing methods for coupled neutronics and 
                thermal hydraulics in fluid fuelled reactors
        \item modeling collaborations between \gls{ARFC} and neutron transport 
                or thermal hydraulics theorists 
        \item algorithmic and data science collaborations on accelerating monte 
                carlo neutron transport calculations with machine learning or 
                information theoretic technicques
\end{itemize}







\bibliographystyle{unsrt}
\bibliography{2017-inl-ldrd}


\end{document}

