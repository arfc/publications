\documentclass[11pt]{article}

\usepackage[acronym,toc]{glossaries}
\include{acros}
\makeglossaries
\usepackage{fancyhdr}
\usepackage{pagecounting}
\usepackage[dvips]{color}
\usepackage{graphicx}
\usepackage{tikz}
\usetikzlibrary{arrows,positioning}
% Color Information from - http://www-h.eng.cam.ac.uk/help/tpl/textprocessing/latex_advanced/node13.html

% Trying to bold my name in the bib
\usepackage{xstring}
\def\FormatName#1{%
  \IfSubStr{#1}{Huff}{\textbf{#1}}{#1}%
}

\usepackage[left=1in, right=1in, top=1in, bottom=1in]{geometry}
\newcommand\bb[1]{\mbox{\em #1}}
\def\baselinestretch{1.1}
%\pagestyle{empty}
\newcommand{\hsp}{\hspace*{\parindent}}
\definecolor{gray}{rgb}{0.4,0.4,0.4}

\newcommand{\authorname}{Kathryn~D.~Huff }
\newcommand{\authoremail}{katyhuff@gmail.com}
\newcommand{\authorsite}{arfc.npre.illinois.edu}

\begin{document}

\pagestyle{fancy}
%\pagenumbering{gobble}
%\fancyhead[location]{text}
% Leave Left and Right Header empty.
%\lhead{}
%\rhead{}
\lhead{\textcolor{gray}{Investigator: Prof. \authorname}}
\rhead{\textcolor{gray}{Advanced Reactors and Fuel Cycles}}
%\rhead{\textcolor{gray}{\thepage/\totalpages{}}}
\renewcommand{\headrulewidth}{0pt}
\renewcommand{\footrulewidth}{0pt}
\fancyfoot[C]{\footnotesize \textcolor{gray}{\authorsite}}

\paragraph{The \gls{ARFC} group} pursues improved 
safety and sustainability of nuclear power. To do so, we model and simulate novel reactor designs
boasting inherent safety features (i.e. accident tolerant fuels or non-voiding 
coolants) and sustainable fuel cycles (i.e. high fuel utilization, online 
reprocessing strategies). These analyses rely on developing computational 
methods that navigate trade-offs between accuracy and compute time, with 
special treatment of coupled feedbacks among thermal-hydraulic phenomena, 
neutron transport, and fuel performance.
Simulations which faithfully capture this coupling at realistic spatial and
temporal resolution are only possible with the aid of high performance
computing resources such as the Blue Waters supercomputing facility which this 
group relies on.

\paragraph{The state of the art} in advanced nuclear reactor simulation (e.g. the 
CASL DOE innovation hub) is focused on traditional light water reactors.  The 
\gls{ARFC} group seeks to extend that state of the art by enabling similarly high 
fidelity simulation of more advanced reactor designs through development of 
models and tools applicable to their unique materials, geometries, and physics. 


\paragraph{Our current work} models the transient, three-dimensional, multi-scale physics
of promising molten salt reactors such as the \gls{MSRE}, \gls{MSBR}, and \gls{FHR} designs.  
This work is well underway with our development of the Moltres application 
\cite{lindsay_moltres_2017}, which relies on the \gls{MOOSE} framework \cite{gaston_moose:_2009}. 
Initial steady state simulations of the \gls{MSRE} have leveraged the Blue Waters 
supercomputer to establish successful coupling between neutronics and thermal 
hydraulics using the \gls{ARFC}-developed Moltres tool. Validation and 
verification efforts are ongoing and upcoming transient simulations seek to explore 
parameters that might increase \gls{LOFC}, \gls{LOHS}, and \gls{RIA} 
survivability in novel reactors.

\paragraph{Potential areas of future collaboration include:}
\begin{itemize}
        \item validation collaborations between \gls{ARFC} and materials scientists, especially 
                experimentalists interested in high flux, high heat performance of 
                molten fluoride salts, liquid metals, graphites, silicon 
                carbides, and uranium nitrides
        \item verification benchmarking collaborations between \gls{ARFC} and computationalists 
                interested in comparing methods for coupled neutronics and 
                thermal hydraulics in fluid fuelled reactors
        \item modeling collaborations between \gls{ARFC} and neutron transport 
                or thermal hydraulics theorists 
        \item algorithmic and data science collaborations on accelerating monte 
                carlo neutron transport calculations with machine learning or 
                information theoretic technicques
\end{itemize}

\paragraph{Our future work} will approach similarly challenging materials and geometries such as those in sodium cooled, gas cooled, and very high temperature reactor designs which also boast improved safety and sustainability.  






\bibliographystyle{unsrt}
\bibliography{2017-zjui-prop}


\end{document}

