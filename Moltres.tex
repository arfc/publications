\documentclass{article}
\setlength{\parindent}{0pt}
\setlength{\parskip}{2ex plus 0.5ex minus 0.2ex}
\usepackage[margin=1in]{geometry}
\renewcommand{\topfraction}{0.9}
\renewcommand{\bottomfraction}{0.8}
\setcounter{topnumber}{2}
\setcounter{bottomnumber}{2}
\setcounter{totalnumber}{4}
\renewcommand{\textfraction}{0.07}
\renewcommand{\floatpagefraction}{0.7}
\usepackage{graphicx}
\usepackage{textcomp}
\usepackage{placeins}
\usepackage[T1]{fontenc}
\usepackage{gensymb}
\usepackage[utf8]{inputenc}
\usepackage{caption}
\usepackage[export]{adjustbox}
\graphicspath{{./figures/}}
% hyperref usually has to go last
\usepackage[hidelinks]{hyperref}
% but glossaries behaves best if after hyperref
\usepackage[acronym,toc]{glossaries}
\include{acros}
\makeglossaries
% cleveref only behaves if after hyperref & glossaries
\usepackage{cleveref}

\let\Oldsection\section
\renewcommand{\section}{\FloatBarrier\Oldsection}

\let\Oldsubsection\subsection
\renewcommand{\subsection}{\FloatBarrier\Oldsubsection}

\let\Oldsubsubsection\subsubsection
\renewcommand{\subsubsection}{\FloatBarrier\Oldsubsubsection}

\newcommand{\code}[1]{\texttt{#1}}

\title{Introduction to Moltres: an Application for Simulation of Molten Salt Reactors}
\author{Alexander Lindsay, Kathryn Huff}

\begin{document}
\maketitle

\section{Governing Equations}

Neutrons are described with time-dependent multi-group diffusion theory as shown
in \cref{eq:neutrons}:

\begin{equation}
%% \frac{1}{v_g}\frac{\partial \phi_g}{\partial t} - \nabla \cdot D_g \nabla \phi_g
%% + \Sigma_g^r \phi_g = \sum_{g \ne g'}^G \Sigma_{g'\rightarrow g}^s \phi_{g'} + \chi_g^p \sum_{g' = 1}^G (1 - \beta)
%% \nu \Sigma_{g'}^f \phi_{g'}
\frac{1}{v_g}\frac{\partial \phi_g}{\partial t} - \nabla \cdot D_g \nabla \phi_g
+ \Sigma_g^r \phi_g = \sum_{g \ne g'}^G \Sigma_{g'\rightarrow g}^s \phi_{g'} + \chi_g^p \sum_{g' = 1}^G (1 - \beta)
\nu \Sigma_{g'}^f \phi_{g'} + \chi_g^d \sum_i^I \lambda_i C_i
\label{eq:neutrons}
\end{equation}

Delayed neutron precursors are described by \cref{eq:precursors}:

\begin{equation}
\frac{\partial C_i}{\partial t} = \sum_{g'= 1}^G \beta_i \nu \Sigma_{g'}^f
\phi_{g'} - \lambda_i C_i - \frac{\partial}{\partial z} u C_i
\label{eq:precursors}
\end{equation}

with the last term representing the effect of fuel advection. The governing
equation for the temperature is given by:

\begin{equation}
  \rho_fc_{p,f}\frac{\partial T_f}{\partial t} + \nabla\cdot\left(\rho_f c_{p,f}
  \vec{u}\cdot T_f -k_f\nabla T_f\right) =  Q_f
  \label{eq:fuel_temp}
\end{equation}

in the fuel and by:

\begin{equation}
  \rho_gc_{p,g}\frac{\partial T_g}{\partial t} + \nabla\cdot\left(-k_g\nabla T_g\right) =  Q_g
  \label{eq:moderator_temp}
\end{equation}

in the moderator. $Q_f$ is defined by:

\begin{equation}
  Q_f = \sum_{g=1}^G \epsilon_{f,g}\Sigma_{f,g}\phi_g
  \label{eq:fuel_source}
\end{equation}

and $Q_g$ is equal to $\gamma Q_f$, representing heat dissipation by gamma and
neutron irradiation in the moderator.

Group constants are generated with either Serpent \cite{leppanen_serpent_2015}
or Scale \cite{dehart_reactor_2011}. Temperature vs group constant interpolation
tables are constructed separately for fuel and moderator regions.

\section{Results \& Discussion}

\subsection{Continuous Galerkin for Temperature with Isotropic Stabilization}

\Cref{fig:cg_group1,fig:cg_group2} show expected cosinusoidal shapes for the
fast and thermal group fluxes with respect to the axial coordinate. Fast group neutrons
show a slight preference for the fuel region, while the thermal group slightly
prefers the moderator as we would intuitively expect.

%% \begin{figure}[H]
\begin{figure}[htpb]
  \centering
  \includegraphics[width=.5\textwidth]{nt_group_1.png}
  \caption{Fast group neutron flux for Continuous Galerkin discretization}
  \label{fig:cg_group1}
\end{figure}

%% \begin{figure}[H]
\begin{figure}[htpb]
  \centering
  \includegraphics[width=.5\textwidth]{nt_group_2.png}
  \caption{Thermal group neutron flux for Continuous Galerkin discretization}
  \label{fig:cg_group2}
\end{figure}

In \cref{fig:cg_temperature} we see that the reactor outlet temperature is
around 992 K, which is within the realm of physical possiblity. The temperature
monotonically increases with increasing axial coordinate in the direction of
flow; radial profiles are uniform.

%% \begin{figure}[H]
\begin{figure}[htpb]
  \centering
  \includegraphics[width=.5\textwidth]{temperature.png}
  \caption{Temperature for Continuous Galerkin discretization}
  \label{fig:cg_temperature}
\end{figure}

\Cref{fig:cg_longest_precursor,fig:cg_shortest_precursor} show the profiles of
the longest and shortest lived precursors respectively. The longest lived
precursor concentration peaks at the reactor outlet, while the shortest lived
precursor shows a maximum concentration at approximately z = 85 cm. The use of
constant monomial basis functions for the precursor solution approximation is
evident in the elemental visualizations.

%% \begin{figure}[H]
\begin{figure}[htpb]
  \centering
  \includegraphics[width=.5\textwidth]{longest_lived_precursor.png}
  \caption{Longest lived precursor for Continuous Galerkin discretization}
  \label{fig:cg_longest_precursor}
\end{figure}

%% \begin{figure}[H]
\begin{figure}[htpb]
  \centering
  \includegraphics[width=.5\textwidth]{shortest_lived_precursor.png}
  \caption{Shortest lived precursor for Continuous Galerkin discretization}
  \label{fig:cg_shortest_precursor}
\end{figure}

\FloatBarrier

\subsection{Discontinuous Galerkin for Temperature}

Moving to a discontinuous Galerkin discretization and removing isotropic
stabilization yields a dramatically different temperature profile, as shown in
\cref{fig:dg_temperature}. With dramatically reduced heat flux due to removal of
artificial conduction, the outlet temperature of the reactor is now 1119 K,
about 130 K higher than the \gls{CG} case shown in \cref{fig:cg_temperature}.

%% \begin{figure}[H]
\begin{figure}[htpb]
  \centering
  \includegraphics[width=.5\textwidth]{dg_temperature.png}
  \caption{Temperature for Discontinuous Galerkin discretization}
  \label{fig:dg_temperature}
\end{figure}

As shown in \cref{fig:dg_group1,fig:dg_group2}, the shapes of the neutron flux
distributions do not change with the move to \gls{DG}. The magnitudes of the
fluxes do increase by about 35\%.

%% \begin{figure}[H]
\begin{figure}[htpb]
  \centering
  \includegraphics[width=.5\textwidth]{dg_group1.png}
  \caption{Fast group neutron flux for Discontinuous Galerkin discretization}
  \label{fig:dg_group1}
\end{figure}

%% \begin{figure}[H]
\begin{figure}[htpb]
  \centering
  \includegraphics[width=.5\textwidth]{dg_group2.png}
  \caption{Thermal group neutron flux for Discontinuous Galerkin discretization}
  \label{fig:dg_group2}
\end{figure}

\FloatBarrier

\subsection{DG Temperature with Moderator Heat Source}

The \gls{MSR} model can be modified by adding a heat source in the moderator
corresponding to gamma heating and neutron irradiation \cite{robertson_conceptual_1971}.

Introduction of the moderator heat source further modifies the reactor
temperature profile as shown in
\cref{fig:dg_mod_source_temperature,fig:dg_mod_source_temperature_outlet}. Without
the moderator heat source, the maximum temperature in the reactor for \gls{CG}
and \gls{DG} discretizations is 992 and 1119 K respectively. These maxima both
occur at the reactor outlet and are more or less uniform over the radial
coordinate. However, with the addition of the moderator heat source, the reactor
temperature maximum rises to a somewhat absurd 1615 K (this is only 88 K below
the boiling point of \gls{FLiBe}). The location of the maximum is in the top right
corner of the simulation domain, corresponding to the middle of the graphite
moderator in a periodic lattice arrangement.

%% \begin{figure}[H]
\begin{figure}[htpb]
  \centering
  \includegraphics[width=.5\textwidth]{dg_mod_source_temperature_full_domain.png}
  \caption{Temperature for Discontinuous Galerkin discretization with moderator
    heat source}
  \label{fig:dg_mod_source_temperature}
\end{figure}

The temperature profile at the top of the reactor is no longer uniform with
respect to the radial coordinate as shown in
\cref{fig:dg_mod_source_temperature_outlet}. From the right edge of the
simulation domain (r = R$_2$), the graphite temperature decreases from its
maximum value of 1615 K to a value of about 1200 K at the fuel/graphite
interface. At the interface, perhaps due to the \gls{DG} discretization and the
order of magnitude difference in diffusion coefficients, there is a shock and
temperature oscillations in the fuel. After the oscillations are damped, the
fuel outlet temperature is uniform between r = 0 and 1.2 cm at a value of
approximately 990 K. This corresponds fairly closely to the outlet temperature
calculated in the isotropically stabilized \gls{CG} simulation, but the
correspondence is believed to be purely coincidental.

%% \begin{figure}[H]
\begin{figure}[htpb]
  \centering
  \includegraphics[width=.5\textwidth]{dg_mod_source_temperature_outlet.png}
  \caption{Fuel outlet temperature profile for Discontinuous Galerkin
    discretization with moderator heat source}
  \label{fig:dg_mod_source_temperature_outlet}
\end{figure}

\Cref{fig:dg_mod_source_group1,fig:dg_mod_source_group2} show the neutron fluxes
for the moderator heat source simulation case. Spatial shapes are the same as
the previous two simulation cases; howevever, the flux magnitudes are
less. A comparison of maximum flux magnitudes between the three simulation cases
is shown in \cref{tab:max_fluxes}.

%% \begin{figure}[H]
\begin{figure}[htpb]
  \centering
  \includegraphics[width=.5\textwidth]{dg_mod_source_group1.png}
  \caption{Fast group neutron flux for Discontinuous Galerkin discretization
    with moderator heat source}
  \label{fig:dg_mod_source_group1}
\end{figure}

%% \begin{figure}[H]
\begin{figure}[htpb]
  \centering
  \includegraphics[width=.5\textwidth]{dg_mod_source_group2.png}
  \caption{Thermal group neutron flux for Discontinuous Galerkin discretization
    with moderator heat source}
  \label{fig:dg_mod_source_group2}
\end{figure}

\begin{table}[htpb]
    \begin{center}
      \begin{tabular}{l|c|c}
        Simulation Case & Maximum Fast Flux ($10^{15}\frac{\#}{cm^2s}$) & Maximum Thermal
        Flux ($10^{15}\frac{\#}{cm^2s}$)\\
        \hline \hline
        \gls{CG} & 3.488 & 1.785 \\
        \gls{DG} & 4.700 & 2.465 \\
        \gls{DG} with moderator source & 2.751 & 1.408 \\
      \end{tabular}
    \end{center}
    \caption{Maximum fluxes for different simulation cases}
    \label{tab:max_fluxes}
\end{table}

\FloatBarrier
\clearpage
\printglossary[type=\acronymtype]
\bibliographystyle{unsrt}
\bibliography{Moltres}
\end{document}
