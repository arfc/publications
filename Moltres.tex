\documentclass{article}
\setlength{\parindent}{0pt}
\setlength{\parskip}{2ex plus 0.5ex minus 0.2ex}
\usepackage[margin=1in]{geometry}
\renewcommand{\topfraction}{0.9}
\renewcommand{\bottomfraction}{0.8}
\setcounter{topnumber}{2}
\setcounter{bottomnumber}{2}
\setcounter{totalnumber}{4}
\renewcommand{\textfraction}{0.07}
\renewcommand{\floatpagefraction}{0.7}
\usepackage{graphicx}
\usepackage{textcomp}
\usepackage{placeins}
\usepackage[T1]{fontenc}
\usepackage{gensymb}
\usepackage[utf8]{inputenc}
\usepackage{caption}
\usepackage[export]{adjustbox}
\graphicspath{{./figures/}}
% hyperref usually has to go last
\usepackage[hidelinks]{hyperref}
% but glossaries behaves best if after hyperref
\usepackage[acronym,toc]{glossaries}
\include{acros}
\makeglossaries
% cleveref only behaves if after hyperref & glossaries
\usepackage{cleveref}

\let\Oldsection\section
\renewcommand{\section}{\FloatBarrier\Oldsection}

\let\Oldsubsection\subsection
\renewcommand{\subsection}{\FloatBarrier\Oldsubsection}

\let\Oldsubsubsection\subsubsection
\renewcommand{\subsubsection}{\FloatBarrier\Oldsubsubsection}

\newcommand{\code}[1]{\texttt{#1}}

\title{Introduction to Moltres: an Application for Simulation of Molten Salt Reactors}
\author{Alexander Lindsay, Kathryn Huff}

\begin{document}
\maketitle

\section{Governing Equations}

Neutrons are described with time-dependent multi-group diffusion theory as shown
in \cref{eq:neutrons}:

\begin{equation}
%% \frac{1}{v_g}\frac{\partial \phi_g}{\partial t} - \nabla \cdot D_g \nabla \phi_g
%% + \Sigma_g^r \phi_g = \sum_{g \ne g'}^G \Sigma_{g'\rightarrow g}^s \phi_{g'} + \chi_g^p \sum_{g' = 1}^G (1 - \beta)
%% \nu \Sigma_{g'}^f \phi_{g'}
\frac{1}{v_g}\frac{\partial \phi_g}{\partial t} - \nabla \cdot D_g \nabla \phi_g
+ \Sigma_g^r \phi_g = \sum_{g \ne g'}^G \Sigma_{g'\rightarrow g}^s \phi_{g'} + \chi_g^p \sum_{g' = 1}^G (1 - \beta)
\nu \Sigma_{g'}^f \phi_{g'} + \chi_g^d \sum_i^I \lambda_i C_i
\label{eq:neutrons}
\end{equation}

Delayed neutron precursors are described by \cref{eq:precursors}:

\begin{equation}
\frac{\partial C_i}{\partial t} = \sum_{g'= 1}^G \beta_i \nu \Sigma_{g'}^f
\phi_{g'} - \lambda_i C_i - \frac{\partial}{\partial z} u C_i
\label{eq:precursors}
\end{equation}

with the last term representing the effect of fuel advection. The governing
equation for the temperature is given by:

\begin{equation}
  \rho_fc_{p,f}\frac{\partial T_f}{\partial t} + \nabla\cdot\left(\rho_f c_{p,f}
  \vec{u}\cdot T_f -k_f\nabla T_f\right) =  Q_f
  \label{eq:fuel_temp}
\end{equation}

in the fuel and by:

\begin{equation}
  \rho_gc_{p,g}\frac{\partial T_g}{\partial t} + \nabla\cdot\left(-k_g\nabla T_g\right) =  Q_g
  \label{eq:moderator_temp}
\end{equation}

in the moderator. $Q_f$ is defined by:

\begin{equation}
  Q_f = \sum_{g=1}^G \epsilon_{f,g}\Sigma_{f,g}\phi_g
  \label{eq:fuel_source}
\end{equation}

and $Q_g$ is equal to $\gamma Q_f$, representing heat dissipation by gamma and
neutron irradiation in the moderator.

Group constants are generated with either Serpent \cite{leppanen_serpent_2015}
or Scale \cite{dehart_reactor_2011}. Temperature vs group constant interpolation
tables are constructed separately for fuel and moderator regions.

\section{Results \& Discussion}


\FloatBarrier
\clearpage
\printglossary[type=\acronymtype]
\bibliographystyle{unsrt}
\bibliography{Moltres}
\end{document}
