%        File: siip.tex
%     Created: Mon Feb 27 08:00 AM 2017 C
% Last Change: Mon Feb 27 08:00 AM 2017 C
%
\documentclass[11pt]{article}

\usepackage[acronym,toc]{glossaries}
\include{acros}
\makeglossaries
\usepackage{fancyhdr}
\usepackage{pagecounting}
\usepackage[dvips]{color}
\usepackage{graphicx}
\usepackage{caption}
\usepackage{subcaption}

% Trying to bold my name in the bib
\usepackage{xstring}
\def\FormatName#1{%
          \IfSubStr{#1}{Huff}{\textbf{#1}}{#1}%
          }

          \usepackage[left=1in, right=1in, top=1in, bottom=1in]{geometry}
          \newcommand\bb[1]{\mbox{\em #1}}
          \def\baselinestretch{1.1}
          %\pagestyle{empty}
          \newcommand{\hsp}{\hspace*{\parindent}}
          \definecolor{gray}{rgb}{0.4,0.4,0.4}

          \newcommand{\authorname}{Kathryn~D.~Huff }
          \newcommand{\authoremail}{katyhuff@illinois.edu}
          \newcommand{\authorsite}{arfc.npre.illinois.edu}

          \begin{document}

          \pagestyle{fancy}
          %\pagenumbering{gobble}
          %\fancyhead[location]{text}
          % Leave Left and Right Header empty.
          %\lhead{}
          %\rhead{}
          \lhead{\textcolor{gray}{Investigator: Prof. \authorname 
          <\authoremail>}}
          \rhead{\textcolor{gray}{SIIP Pre-Proposal\\Dept. of Nuclear, 
          Plasma, and Radiological Engineering}}
          %\rhead{\textcolor{gray}{\thepage/\totalpages{}}}
          \renewcommand{\headrulewidth}{0pt}
          \renewcommand{\footrulewidth}{0pt}
          \fancyfoot[C]{\footnotesize \textcolor{gray}{\authorsite}}

          \section{The Problem}
          As you read this, thousands of professors in the United States are 
          simultaneously preparing lessons on transposing a matrix.
          They are doing so largely without recieving feedback from one another 
          or directly building on one another's experience 
          \cite{wilson_software_2014}. In this way, professors spend an enormous amount of time 
          duplicating curriculum development efforts already tackled 
          by colleagues. What's worse is that these efforts are 
          rarely, if ever, reviewed by, shared with, or extended upon by 
          peers.
          
          Wikipedia and open source software development provide excellent 
          examples of distributed expert collaboration and dynamic peer review 
          on a master product. So, why aren't professors sharing their lesson 
          materials online, collaborating on canonical lesson sets, 
          diffing and merging similar lessons, and reviewing one another's 
          work? \emph{Our goal is to explore the possibility that curriculum development for 
          university courses can operate like open source software development 
          does.}

          \section{The Proposed Solution}
          We propose a proof-of-concept for collaborative, open source, 
          curriculum development to improve the transfer of lessons learned 
          between instructors of the same course (either at a single university or 
          among different campuses). This small-scale prototype collaboration 
          will provide a framework which could be adopted for collaboration among faculty 
          teaching courses with an inherently larger scale (e.g. CS101).
          
          
          This initiative already involves a few peer professors. Six faculty 
          in Nuclear Engineering from UC Berkeley, the University of Wisconsin 
          at Madison, Kansas State, U. South Carolina, and U. 
          Tennessee-Knoxville are involved. We all use GitHub \cite{github} to 
          store, revise, and collaborate on research, especially source code. A 
          few of us have already started to host our course curricula online as 
          well, but these are typically single author repositories 
          \cite{skutnik,wilson,slaybaugh,scopatz,borelli,roberts}.

          \paragraph{In a typical open source project,} a main copy of the 
          repository (or ``fork'') holds the 
          official copy of the software package. Individual developers each have their own 
          forks where they can work on features and 
          bug fixes. When the developer makes changes that are ready for prime 
          time, they make a ``pull request'' to the main fork, it is 
          reviewed by their collaborators, and it is eventually merged into the 
          main fork where it can be used by all. Figures 
          \ref{fig:sub1} and \ref{fig:sub2} show how the workflow used for many 
          open source projects can be leveraged toward course development.

          \begin{figure}
                  \centering
                  \begin{subfigure}{.4\textwidth}
                            \centering
                              \includegraphics[width=.8\linewidth]{git-flow}
        \caption{This figure, from \cite{scopatz_effective_2016}, captures a process, often called Git Flow, through which a new feature or bug fix enters a piece of open source software \cite{scopatz_effective_2016}.}
                                  \label{fig:sub1}
                  \end{subfigure}\hfill%
                  \begin{subfigure}{.4\textwidth}
                            \centering
                              \includegraphics[width=.8\linewidth]{siip-flow}
  \caption{This adaptation imagines the process in the context of learning module development, using the same git version control system and GitHub repository hosting framework.}
                                  \label{fig:sub2}
                  \end{subfigure}
                  \label{fig:test}
          \end{figure}

          Such projects occur drawing on a community of experts toward 
          creating a robust shared resource. In open source software, 
          developers share code revisions in online repositories, review one 
          anothers work, and contribute back their own improvements to the main 
          project. 


          \paragraph{The participants will collaborate} on a master set of learning 
          modules for an upper-division course in nuclear engineering : 
          The Nuclear Fuel Cycle. In the near term, we will develop fine 
          grained learning modules which may be mixed and matched to meet the 
          learning . The curriculum will be hosted on GitHub, 
          test by all of us, and improved continually as we learn and grow as 
          instructors. 

          \paragraph{Each learning module} may include active in-class exercises, 
          presentation notes, homework problems, project descriptions, and 
          evaluation tools. 

          \paragraph{This challenge has been encountered} by collaborative 
          instruction organizations already. Most notable among these is 
          Software Carpentry, in which experience shows 
          \cite{wilson_lessons_2014} forks of the original material tend to diverge 
          without any strong incentive for contribution back to the master 
          material \cite{wilson_software_2014}.


          \section{The Potential Impact}
          \cite{wilson_software_2014}.

          Doing code review at the end of the work isn't useful. What works is 
          incremental code review. This proposal suspects that the same is true 
          for curriculum review. 
          \cite{wilson_software_2014}.

          Education is an inherently distributed system, but this need not be a 
          hinderance to collaboration.

          \paragraph{We expect that} fine-grained modularity of lesson components and a clear dependency graph of prerequisite modules for each lesson may assist in overcoming this goal. Further, a small 
          and devoted group of like-minded nuclear engineering professors 
          should be able to compromise on a cohesive set of modular materials 
          that can be mixed and matched according to individual preference. 
          
          This work will provide an important proof of concept for groups of 
          instructors willing to collaborate on open source curriculum for:
          \begin{itemize}
                  \item core courses with many sections in a single university
                  \item niche courses taught by a select group of professors across 
          universities
                  \item fundamental courses in small fields (e.g. nuclear engineering)
          \end{itemize}

          Through exploration of the option space, we will arrive at a raw 
          curriculum format for each type of learning module.  

          Scientists are more than happy to build upon one another's work, form 
          collaborations with others in their field. But, 
          when it comes to educating, where is the sharing of lessons learned 
          and collaboration? 

          \bibliographystyle{katyunsrt}
          \bibliography{bibliography}

          \section{Departmental Support}
          The Department of Nuclear, Plasma, and Radiological Engineering will 
          support this work with relese time through startup funds. 
          Additionally, NPRE will support workshop activities through 
          administrative effort and by providing space.
          \end{document}


