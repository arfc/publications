\documentclass[11pt]{article}

\usepackage[acronym,toc]{glossaries}
\include{acros}
\makeglossaries
\usepackage{fancyhdr}
\usepackage{pagecounting}
\usepackage[dvips]{color}
\usepackage{graphicx}
\usepackage{tikz}
\usetikzlibrary{arrows,positioning}
% Color Information from - http://www-h.eng.cam.ac.uk/help/tpl/textprocessing/latex_advanced/node13.html

% Trying to bold my name in the bib
\usepackage{xstring}
\def\FormatName#1{%
  \IfSubStr{#1}{Huff}{\textbf{#1}}{#1}%
}

\usepackage[left=1in, right=1in, top=1in, bottom=1in]{geometry}
\newcommand\bb[1]{\mbox{\em #1}}
\def\baselinestretch{1.1}
%\pagestyle{empty}
\newcommand{\hsp}{\hspace*{\parindent}}
\definecolor{gray}{rgb}{0.4,0.4,0.4}

\newcommand{\authorname}{Kathryn~D.~Huff }
\newcommand{\authoremail}{katyhuff@gmail.com}
\newcommand{\authorsite}{arfc.npre.illinois.edu}

\begin{document}

\pagestyle{fancy}
%\pagenumbering{gobble}
%\fancyhead[location]{text}
% Leave Left and Right Header empty.
%\lhead{}
%\rhead{}
\lhead{\textcolor{gray}{Investigator: Prof. \authorname\\Presenter: Mr. Andrei Rykhlevskii}}
\rhead{\textcolor{gray}{University of Illinois at Urbana-Champaign\\}}
%\rhead{\textcolor{gray}{\thepage/\totalpages{}}}
\renewcommand{\headrulewidth}{0pt}
\renewcommand{\footrulewidth}{0pt}
\fancyfoot[C]{\footnotesize \textcolor{gray}{\authorsite}}
   \begin{center}
      \Large\textbf{Computational tools for advanced Molten Salt Reactors simulation}\\
   \end{center}

The Advanced Reactors and Fuel Cycles (ARFC) group models and simulates the design, safety, and performance of advanced nuclear reactors. Such reactors  involve tight coupling between physics such as thermal-hydraulic phenomena, neutron transport, and fuel performance. Current work introduces an extension of the MOOSE framework, Moltres, to appropriately model coupled thermal-hydraulics and neutronics of promising molten-salt-fueled reactor designs. Additionally, we have developed an online reprocessing simulation tool, SaltProc which includes fission product removal, fissile material separations, and refuelling for time dependent analysis of fuel-salt evolution. 

Initial simulations of the Molten Salt Reactor Experiment and the conceptual 
Molten Salt Breeder Reactor have been conducted on Blue Waters with deterministic multiphysics and Monte Carlo methods respectively. Steady state, transient, and fuel cycle analysis simulations have been run in 2D as well as 3D and compared against the Molten Salt Reactor Experiment. Fuel cycle dynamics and quasi-equilibrium compositions were obtained from depletion and reprocessing simulations for a 20-year time frame. The MSBR full-core safety analysis was performed at the startup and equilibrium fuel salt compositions, for important reactor safety parameters. In this work, conducted on Blue Waters, ARFC has demonstrated the capability to model complex physics in an advanced molten-salt-fueled nuclear reactor.

\end{document}
