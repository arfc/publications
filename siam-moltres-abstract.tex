\documentclass[11pt]{article}

\usepackage[acronym,toc]{glossaries}
\include{acros}
\makeglossaries
\usepackage{fancyhdr}
\usepackage{pagecounting}
\usepackage[dvips]{color}
\usepackage{graphicx}
\usepackage{tikz}
\usetikzlibrary{arrows,positioning}
% Color Information from - http://www-h.eng.cam.ac.uk/help/tpl/textprocessing/latex_advanced/node13.html

% Trying to bold my name in the bib
\usepackage{xstring}
\def\FormatName#1{%
  \IfSubStr{#1}{Huff}{\textbf{#1}}{#1}%
}

\usepackage[left=1in, right=1in, top=1in, bottom=1in]{geometry}
\newcommand\bb[1]{\mbox{\em #1}}
\def\baselinestretch{1.1}
%\pagestyle{empty}
\newcommand{\hsp}{\hspace*{\parindent}}
\definecolor{gray}{rgb}{0.4,0.4,0.4}

\newcommand{\authorname}{Kathryn~D.~Huff }
\newcommand{\authoremail}{katyhuff@gmail.com}
\newcommand{\authorsite}{arfc.npre.illinois.edu}

\begin{document}

\pagestyle{fancy}
%\pagenumbering{gobble}
%\fancyhead[location]{text}
% Leave Left and Right Header empty.
%\lhead{}
%\rhead{}
\lhead{\textcolor{gray}{Investigator: Prof. \authorname\\Presenter: Mr. Andrei Rykhlevskii}}
\rhead{\textcolor{gray}{University of Illinois at Urbana-Champaign\\}}
%\rhead{\textcolor{gray}{\thepage/\totalpages{}}}
\renewcommand{\headrulewidth}{0pt}
\renewcommand{\footrulewidth}{0pt}
\fancyfoot[C]{\footnotesize \textcolor{gray}{\authorsite}}
   \begin{center}
      \Large\textbf{Simulation of Molten Salt Reactors with Moltres}\\
   \end{center}

The Advanced Reactors and Fuel Cycles (ARFC) group models and simulates the design, safety, and performance of advanced nuclear reactors. For these simulations coupling between physics such as neutron transport, thermal-hydraulics phenomena, and fuel performance must be taken into account. Our group performs high fidelity simulation of Gen IV reactor designs through development of models and tools for representing unique materials, complex geometries, and physical phenomena. Current work introduces an extension of the MOOSE framework, Moltres, to appropriately model coupled thermal-hydraulics and neutronics of promising liquid-fueled Molten Salt Reactor designs. 

Initial simulations of the Molten Salt Reactor Experiment (MSRE) have been conducted on Blue Waters supercomputer with deterministic multiphysics. Steady state, transient, and fuel cycle analysis simulations have been run in 2D as well as 3D and compared against the Molten Salt Reactor Experiment. These simulations
have occupied up to many hundreds of nodes simultaneously and have resulted in rich datasets
for use in reactor design and analysis. This talk will describe how coupling between neutronics and thermal hydraulics have been established in the Moltres as well as the validation and verification efforts which have been completed. 

\end{document}
